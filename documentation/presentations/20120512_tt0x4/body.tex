
\newlength{\smallcol}
\setlength{\smallcol}{0.3\textwidth}

\newlength{\bigcol}
\setlength{\bigcol}{\textwidth}
\addtolength{\bigcol}{- \smallcol}


\begin{frame}[plain]
  \mode<article>{\maketitle}
  \mode<presentation>{\titlepage}
\end{frame}

\section{Status quo}

\subsection{Specification of physical interface between modules}
  \begin{frame}[<.->]{Specification of physical interface between modules}
  	\begin{columns}
    	\begin{column}{\smallcol}
 				\begin{center}\includegraphics<1->[width=\textwidth]{friendship0}\end{center}
			\end{column}
  	  \begin{column}{\bigcol}
				\begin{itemize}
					\item<+-> Modules are connected via a backplane
					\item<+-> PCIe 4x plug w/ custom pinout
					\item<+-> 2x RS485 lanes for inter-module communication
					\item<+-> SPI-ish time broadcast bus
					\item<+-> Differential clock signal for high-res timing signal
					\item<+-> Each module sports storage for calibration data
				\end{itemize}
			\end{column}
  	\end{columns}
	\end{frame}

\subsection{friendship0 backplane}
	\begin{frame}[<.->]{friendship0 backplane}
  	\begin{columns}
    	\begin{column}{\smallcol}
 				\begin{center}\includegraphics<1->[width=\textwidth]{friendship0_assembled}\end{center}
 				\begin{center}\includegraphics<1->[width=\textwidth]{friendship0_assembled_bottom}\end{center}
			\end{column}
  	  \begin{column}{\bigcol}
				\begin{itemize}
					\item<+-> Four modules slots, one dedicated to bus master module
					\item<+-> ICs for interrupt handling
					\item<+-> Can be easily scaled up, next step eight or nine slots
				\end{itemize}
			\end{column}
  	\end{columns}
	\end{frame}

\subsection{braeburn0 \& 1 power supply module}
	\begin{frame}[<.->]{braeburn0 \& 1 power supply module}
  	\begin{columns}
    	\begin{column}{\smallcol}
 				\begin{center}\includegraphics<1->[width=\textwidth]{braeburn}\end{center}
 				\begin{center}\includegraphics<1->[width=\textwidth]{braeburn_pcb}\end{center}
			\end{column}
  	  \begin{column}{\bigcol}
				\begin{itemize}
					\item<+-> Single external power source
					\item<+-> All voltages generated on-board, stabilized
					\item<+-> In-system voltage level monitoring
					\item<+-> braeburn1 using PC power supply
				\end{itemize}
			\end{column}
  	\end{columns}
	\end{frame}

\subsection{flutter0 high precision distributed time source module}
	\begin{frame}[<.->]{flutter0 high precision distributed time source module}
  	\begin{columns}
    	\begin{column}{\smallcol}
 				\begin{center}\includegraphics<1->[width=\textwidth]{flutter0_layout}\end{center}
			\end{column}
  	  \begin{column}{\bigcol}
				\begin{itemize}
					\item<+-> Spartan3 FPGA for high-res timing (<100 ns)
					\item<+-> ATmega 168 for lo-res timing (1 s to 1/10th s)
					\item<+-> Low cost GPS module w/ external antenna support
				\end{itemize}
			\end{column}
  	\end{columns}
	\end{frame}

\subsection{dash0 proof of concept}
	\begin{frame}[<.->]{dash0 proof of concept}
  	\begin{columns}
    	\begin{column}{\smallcol}
 				\begin{center}\includegraphics<1->[width=\textwidth]{dashpoc}\end{center}
			\end{column}
  	  \begin{column}{\bigcol}
				\begin{itemize}
					\item<+-> ADS-B receiver based around miniADSB module
					\item<+-> Easily track commercial aircrafts
					\item<+-> Perfect for verifying pseudo ranging algorithms
				\end{itemize}
			\end{column}
  	\end{columns}
	\end{frame}

\section{Next up}	

\subsection{celestia0 bus master module}
	\begin{frame}{celestia0 bus master module}
		\begin{itemize}
			\item Manages interrupt requests by modules
			\item Arbitrates resources
			\item Enumeration of available modules
		\end{itemize}
	\end{frame}

\subsection{dash0 ADSB receiver module}
	\begin{frame}{dash0 ADSB receiver module}
		\begin{itemize}
			\item Built around the proof of concept
			\item Most likely CPLD-based decoding of Manchester-encoded signal
			\item Contributions by Pawel
			\item Perfect to test pseudo-ranging because ADSB signal contains GPS location data already (ground truth)
			\item Your own fligt tracking radar at home?  Hell, yeah!
		\end{itemize}
	\end{frame}

\subsection{magic0 bus protocol}
	\begin{frame}{magic0 bus protocol}
		\begin{itemize}
			\item Protocol spoken between modules and master
			\item Handles data exchange and enumeration
		\end{itemize}
	\end{frame}

\subsection{Testing timing accuracy}
	\begin{frame}{Testing timing accuracy}
		\begin{itemize}
			\item First level test: 2x ground stations w/ flutter module
			\item Second level test: ~5 ground stations w/ flutter module
		\end{itemize}
	\end{frame}

\subsection{Calibration}
	\begin{frame}{Calibration}
		\begin{itemize}
			\item High accuracy measurement requires diligent calibration
			\item Receiver, decoder, communication lags
			\item Phase error
			\item ...
		\end{itemize}
	\end{frame}

\subsection{Deploying 5+ systems}
	\begin{frame}{Deploying 5+ systems}
		\begin{itemize}
			\item Test pseudo ranging and timing
			\item This will decide whether tracking would already work with our timing resolution
			\item If not, timing resolution could be scaled up by factor 10 easily
		\end{itemize}
	\end{frame}

\subsection{Quality tests and review}
	\begin{frame}{Quality tests and review}
		\begin{itemize}
			\item Review everything
			\item Make improvements where necessary
			\item Manufacture pre-series
			\item Hand ground stations out to other hackerspaces and interested subsubsectionies
		\end{itemize}
	\end{frame}

\subsection{More modules}
	\begin{frame}{More modules}
		\begin{itemize}
			\item Arduino module
			\begin{itemize}
				\item Probably the easiest way to prototype
				\item Make it available to an already large community
			\end{itemize}
			\item Environment sensors
			\begin{itemize}
				\item Measure ALL the things
				\item Temperature, humidity, barometric pressure, seismic waves, radiation, tectonic drift, time, wind, ...
			\end{itemize}
		\end{itemize}
	\end{frame}

